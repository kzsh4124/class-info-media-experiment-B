\documentclass{ltjsreport}
% 数理系
\usepackage{amsmath,amsbsy,amssymb}
\usepackage{mathtools}
\usepackage{physics2}
\usepackage{siunitx}

% 画像
\usepackage{graphicx}
% ロゴ関係
\usepackage{bxtexlogo}
\bxtexlogoimport{*,**}
% Hオプションを使いたいので読み込む
\usepackage{here}
% ソースコードを表示するのに読み込む
\usepackage{listings} % 用例: \lstinputlisting[caption=sqrt.cpp,style=c++]{./sqrt.cpp}
% シンタックスハイライトのため
\usepackage{xcolor}
% url挿入のため
\usepackage{xurl}
% 枠のため
\usepackage{ascmac}
% xcolorの色定義
\definecolor{solarized@base03}{HTML}{002B36}
\definecolor{solarized@base02}{HTML}{073642}
\definecolor{solarized@base01}{HTML}{586e75}
\definecolor{solarized@base00}{HTML}{657b83}
\definecolor{solarized@base0}{HTML}{839496}
\definecolor{solarized@base1}{HTML}{93a1a1}
\definecolor{solarized@base2}{HTML}{EEE8D5}
\definecolor{solarized@base3}{HTML}{FDF6E3}
\definecolor{solarized@yellow}{HTML}{B58900}
\definecolor{solarized@orange}{HTML}{CB4B16}
\definecolor{solarized@red}{HTML}{DC322F}
\definecolor{solarized@magenta}{HTML}{D33682}
\definecolor{solarized@violet}{HTML}{6C71C4}
\definecolor{solarized@blue}{HTML}{268BD2}
\definecolor{solarized@cyan}{HTML}{2AA198}
\definecolor{solarized@green}{HTML}{859900}
% listingsのスタイル定義
\lstdefinestyle{c}{
  language=c,
  numbers=left,
}
\lstdefinestyle{c++}{
  language=c++,
  numbers=left,
}
\lstdefinestyle{python}{
  language=python,
  numbers=left,
}
\lstset{
basicstyle=\small\ttfamily\color{solarized@base00},
rulesepcolor=\color{solarized@base03},
numberstyle=\scriptsize\color{solarized@base01},
keywordstyle=\color{solarized@blue},
stringstyle=\color{solarized@cyan}\ttfamily,
commentstyle=\color{solarized@base01},
emphstyle=\color{solarized@red},
backgroundcolor=\color{solarized@base3},
sensitive=true,
breaklines=true,
breakatwhitespace=true,
framerule=0pt,
frame=l,
showstringspaces=false,
tabsize=2,
basewidth={0.57em,0.52em},
}

\title{情報メディア実験Bレポート}
\author{Kazushi Nakamura}
\date{\today}

\begin{document}
\maketitle



\tableofcontents

\chapter{はじめに}


\section{目的}
本実験では、ライントレーサロボットの制作を通して、電気回路解析、制御工学に関する理論、および実際のソフトウェア、ハードウェアへの応用を学ぶことを目的とする。
ライントレーサロボットは、周囲と明るさの異なる線(一般的には黒色または白色)を追従する非常に単純なロボットであり、最もシンプルな制御方法とキットを用いればロボット製作が初めての小学生でも容易に制作が可能なレベルである。しかし、フルスクラッチで設計し、高速で追従させようとすると必要な知識は広範にわたり、難易度は高くなる。本レポートでは、ロボット製作の過程での学習成果、および、実際のロボットの設計、製作方法、評価についてまとめる。

\section{本レポートの構成}
%TODO: 制御工学の概説を書く
2章では、電気回路解析の基本についての学習成果として、直流回路および交流回路の解析手法、過渡現象、よく用いられる素子系についてをまとめる。
3章は制御工学の基本についての学習成果をまとめる。
2, 3章の内容は一般的なものであるから、読む時間がない場合は飛ばして次の4章から読むことを推奨する。
4章では、実際に走行体に使用したハードウェア設計について部分に分割して示す。
5章では、ソフトウェア設計について、システム構成、制御手法および、部分ごとの実装について示す。
6章では、走行体の評価、振り返り、改善点、感想等について示す。

\chapter{電気回路解析の基本}
\section{概要}

\section{電気回路の導入}

\section{直流回路の解析}

\section{正弦波交流回路の解析}

\section{過渡現象}

\section{回路素子の性質}




\chapter{古典制御の基本}

\section{概要}

\section{制御系の導入}

\section{ラプラス変換と伝達関数}

\section{PID制御}

\section{安定性解析}



\chapter{ハードウェア設計}
\section{ハードウェア概観}

\section{部品表}

\section{ベース}

\section{センサ部}

\section{メイン基板}







\chapter{ソフトウェア設計}
\section{ソフトウェア概観}
%ブロック線図とかこのへんに書く

\section{センサ値の検出と制御}
\section{速度の検出}

\section{モーター制御}

\chapter{評価}







\appendix

\chapter{応用的な電気回路解析}
\section{4端子回路}

\section{ひずみ波交流の解析}

\section{三相交流}

\section{分布定数回路}


\chapter{設計の変遷}
本付録では、最終成果物に至るまでの設計の変遷を示す。
\section{初号機}


\section{センサ値の統合}



\chapter{インシデント、問題解決集}





\end{document}